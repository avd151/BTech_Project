%_____________________________________________________________________________________________ 
% LATEX Template: Department of Comp/IT BTech Project Reports
% Abstract of Report
% Sun Mar 27 10:34:00 IST 2011
%_____________________________________________________________________________________________ 
%\newpage

%This page is to add the scanned copy of plagiarism percentage page signed by your Project Guide


\newpage
%\begin{abstract}
%\addcontentsline{toc}{chapter}{Abstract}	% This makes sure abstract is included in contents.
\begin{center}
\Large \textbf{Abstract}
\end{center}
Route alignment between two separate locations can be a time-consuming and
costly process. Alignment of the route is the process of positioning the centerline of
the route based on a variety of factors. While doing route alignment, various factors,
such as topography of the area, geospatial features, social implications of the plan,
climatic conditions of the area, presence of obstructions and cost of the plan, must
be taken into account. A route alignment that does the least harm to the
environment, is short, and is economically feasible is the most suitable route
alignment. Geographic Information Systems (GIS) and Remote Sensing play a very
crucial role when used together for facilitating the process of route alignment. Deep
Learning techniques have experienced a surge in use in GIS and Remote Sensing
applications in recent years. The aim of this work is to present an effective solution
to the problem of aligning a route between two locations considered for the case
study using GIS, Remote Sensing and Deep Learning/Machine Learning. Each
aspect influencing route alignment will be assigned different weights based on
importance. A function will compute the individual cost of each route under
consideration, after which the ideal path will be determined. In order to make the
decision, an Analytic Hierarchy Process or an Analytic Network Process will be
used. In this process, different algorithms like Bellman Ford Algorithm, A*
Algorithm, Genetic Algorithm, and Dijkshtra's Algorithm etc will be used
according to the requirements.The final result will thus provide the optimal area for
route alignment between two given two locations. \\
\textbf{KEYWORDS : }Route alignment, Geographical Information System, Remote Sensing, Route planning, Route selection, Route Optimization, Optimal Path
%\end{abstract}

%_____________________________________________________________________________________________ 
